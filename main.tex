\documentclass{article}

\usepackage{amsmath}
\usepackage{amsfonts}
\usepackage{amsthm}
\usepackage{physics}
\usepackage{graphicx}
\usepackage{enumitem}
\usepackage{setspace}
\usepackage{tikz}
\usepackage[bottom]{footmisc}
\usepackage{hyperref}
\usepackage{cleveref}

% Hebrew Stuff
\usepackage{polyglossia}
\usepackage{fontspec}
\setmainlanguage{hebrew}
\setotherlanguage{english}
\newfontfamily\hebrewfont[Script=Hebrew]{David CLM}
\rightfootnoterule
\setlist[itemize,1]{label={\fontfamily{cmr}\fontencoding{T1}\selectfont\textbullet}} % Fix itemize bullets
\makeatletter
\def\maketag@@@#1{\hbox{\m@th\normalfont\LRE{$#1$}}} % Fix equation numbering
\def\tagform@#1{\maketag@@@{(\ignorespaces#1\unskip)}}
\makeatother

\setstretch{1.25}
\setcounter{tocdepth}{1}
\DeclareDocumentCommand\vnabla{}{\nabla}

\title{פפס״י - סדנה 2}
\author{יובל גת, עפרי קארו ויאיר תדהר}
\date{2021, סמסטר א׳}
\begin{document}
	\maketitle
	\section{שאלה 2}
	הפירוק יביא:
	\[
		\begin{cases}
			\frac{d^2x}{dt^2}=0 \\
			\frac{d^2y}{dt^2}=-g
		\end{cases}
	\]
	ועל כן נכתוב:
	\[
		\begin{cases}
			\frac{dx}{dt}=v_{0, x} \\
			\frac{dy}{dt}=v_{0, y}-gt \\
			\frac{dv_x}{dt}=0 \\
			\frac{dv_y}{dt}=-g
		\end{cases}
	\]
	על כן מערכת הצעדים שנקבל היא:
	\[
		\begin{cases}
			x_{n+1}=\Delta tv_x+x_n \\
			y_{n+1}=\Delta t(v_{0, y}-gt_n)+y_n \\
			(v_x)_{n+1}=(v_x)_n \\
			(v_y)_{n+1}=-g\Delta t+(v_y)_n
		\end{cases}
	\]
	נעיר כי בחרנו במהירות
	$v=300\left(\frac{m}{s}\right)$,
	לאחר התייעצות עם קמ״ד בחק״ב ים.
	
	להלן מסלול התנועה של הרקטה:
	\begin{center}
		\includegraphics[scale=0.45]{ex2}
	\end{center}
	נשים לב שהרקטה פוגעת, במקרה זה, בנקודה
	$x\approx 9000$.
	\section{שאלה 3}
	בחרנו בפרמטרים
	$C=0.5, A=0.0707(m^3), \rho=1.184(kg/m^3), m=363(kg)$,
	בהתאם לטיל ״עברי״ וההנחה שהרש״ק שלו חרוטי.
	נרשום את משוואת התנועה כך:
	\[
		m\frac{d^2\vec{r}}{dt^2}=\mqty(-CA\rho\sqrt{v_x^2+v_y^2}v_x \\ -mg-CA\rho\sqrt{v_x^2+v_y^2}v_y)
	\]
	ועל כן:
	\[
		\begin{cases}
			x_{n+1}=\Delta tv_x+x_n \\
			y_{n+1}=\Delta tv_y+y_n \\
			(v_x)_{n+1}=(v_x)_n-\Delta t\frac{CA\rho}{m}\sqrt{(v_x)_n^2+(v_y)_n^2}(v_x)_n \\
			(v_y)_{n+1}=(v_y)_n-\Delta tg-\Delta t\frac{CA\rho}{m}\sqrt{(v_x)_n^2+(v_y)_n^2}(v_y)_n
		\end{cases}
	\]
	מסלול הרקטה במקרה זה יראה כך:
	\begin{center}
		\includegraphics[scale=0.45]{ex3_normal.png}
	\end{center}
	במקרה זה הפגיעה מתרחשת בנקודה
	$x\approx 5000$,
	כמעט כמחצית מהמקרה הקודם.
	בנוסף, ניתן לעשות מספר בדיקות שפיות. הראשונה היא המקרה בו יורים את הרקטה במקביל לקרקע, ומקבלים, כמצופה:
	\begin{center}
		\includegraphics[scale=0.45]{ex3_sanity1}
	\end{center}
	נוכל גם לירות את הרקטות במאונך לקרקע, ובמקרה זה נקבל:
	\begin{center}
		\includegraphics[scale=0.45]{ex3_sanity2}
	\end{center}
	נשים לב שמתקבלת פרבולה ש״רוחבה״ קטן מאוד, $10^{-13}$, ולא קו ישר, מהסיבה שלפייתון יש רמת דיוק מסוימת בחישוב ביטויים עם ערכים אי-רציונליים כגון
	$\pi$.
	
	כעת נתבונן בגרף המתאר את מיקום הפגיעה כפונקציה של רמת הדיוק
	$\Delta t$.
	אנו רואים שכאשר
	$\Delta t\to 0$,
	מיקום הפגיעה מתכנס לנקודה ה״אמיתית״:
	\begin{center}
		\includegraphics[scale=0.45]{ex3_dts}
	\end{center}
	\section{שאלה 4}
	להלן הזווית המתאימה לכל מרחק:
	\begin{center}
		\includegraphics[scale=0.45]{theta}
	\end{center}
	\section{שאלה 5}
	להלן גרף יירוט הרקטה:
	\begin{center}
		\includegraphics[scale=0.45]{intercept.png}
	\end{center}
	ניתן גם לשרטט את המרחק כפונקציה של זמן, והנקודה בה המרחק מתאפס היא נקודת היירוט:
	\begin{center}
		\includegraphics[scale=0.45]{dist}
	\end{center}
	בדיקת השפיות מראה שהזווית במקרה המתואר היא
	$\approx130^\circ$,
	כצפוי; אם הרקטות מתנהגות אותו הדבר, המסלולים הם שיקופים זה של זה, ועל כן מצופה מאיתנו לקבל זווית של
	$180^\circ-50^\circ=130^\circ$.
\end{document}